This part of the lab showed us how to properly format and create plots using Python. When plotting using Python, there are many different settings that can be toggled which dictate the output that is your plot. For example, step size, the figure size of your plot, a plot title, labels, and grids are all things that must be defined in the code for your plot. You can also create subplots in one overarching figure. It is also noted that you must include the plot show command or else you will not be able to view your plots. Plots are very modifiable using Python, and we now have a much better idea of how to implement them in our labs.